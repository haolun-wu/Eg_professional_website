% TODO THIS DOC:
% put reviewing for conferences and journals
% if this sucks you can comment it out and release when you’ve done more
% teaching
% include the number of students (a range like 50-100)
% bulleted descriptions


\documentclass[paper=letter,fontsize=11pt]{scrartcl} % KOMA-article class
							
\usepackage[english]{babel}
\usepackage[utf8x]{inputenc}
\usepackage[protrusion=true,expansion=true]{microtype}
\usepackage{amsmath,amsfonts,amsthm}     % Math packages
\usepackage{graphicx}                    % Enable pdflatex
\usepackage[svgnames]{xcolor}            % Colors by their 'svgnames'
\usepackage{geometry}
	%\textheight=700px                    % Saving trees ;-)
%\usepackage{url}
\usepackage[colorlinks=true,
linkcolor=blue,
urlcolor=blue]{hyperref}
\usepackage{float}
\usepackage{etaremune}
\usepackage{wrapfig}
\usepackage{attachfile}

% my added packages
% \usepackage{bibentry}
% \bibliography*
\PassOptionsToPackage{hyphens}{url}\usepackage{hyperref}
\makeatletter
\g@addto@macro{\UrlBreaks}{\UrlOrds}
\makeatother

\frenchspacing              % Better looking spacings after periods
\pagestyle{empty}           % No pagenumbers/headers/footers

%\addtolength{\voffset}{-40pt}
%\addtolength{\textheight}{20pt}

\setlength\topmargin{0pt}
\addtolength\topmargin{-\headheight}
\addtolength\topmargin{-\headsep}
\setlength\oddsidemargin{0pt}
\setlength\textwidth{\paperwidth}
\addtolength\textwidth{-2in}
\setlength\textheight{\paperheight}
%\addtolength\textheight{-3in}
\addtolength\textheight{-2in}
\usepackage{layout}

%%% Custom sectioning}{sectsty package)
%%% ------------------------------------------------------------
\usepackage{sectsty}

\sectionfont{%			            % Change font of \section command
	\usefont{OT1}{phv}{b}{n}%		% bch-b-n: CharterBT-Bold font
	\sectionrule{0pt}{0pt}{-5pt}{3pt}}

%%% Macros
%%% ------------------------------------------------------------
\newlength{\spacebox}
\settowidth{\spacebox}{8888888888}			% Box to align text
\newcommand{\sepspace}{\vspace*{1em}}		% Vertical space macro

\newcommand{\MyName}[1]{ % Name
		\Huge \usefont{OT1}{phv}{b}{n} \hfill #1
		\par \normalsize \normalfont}
		
\newcommand{\MySlogan}[1]{ % Slogan}{optional)
		\large \usefont{OT1}{phv}{m}{n}\hfill \textit{#1}
		\par \normalsize \normalfont}

\newcommand{\NewPart}[2]{\section*{\uppercase{#1} #2}}

\newcommand{\PersonalEntry}[2]{
		\noindent\hangindent=2em\hangafter=0 % Indentation
		\parbox{\spacebox}{        % Box to align text
		\textit{#1}}		       % Entry name}{birth, address, etc.)
		\hspace{1.5em} #2 \par}    % Entry value

\newcommand{\SkillsEntry}[2]{      % Same as \PersonalEntry
		\noindent\hangindent=2em\hangafter=0 % Indentation
		\parbox{\spacebox}{        % Box to align text
		\textit{#1}}			   % Entry name}{birth, address, etc.)
		\hspace{1.5em} #2 \par}    % Entry value	
		
\newcommand{\EducationEntry}[3]{
		\noindent \textbf{#1} \hfill      % Study
		\colorbox{White}{%
			\parbox{10em}{%
			\hfill\color{Black}#2}} \par  % Duration
		\noindent \textit{#3} \par}
		
\newcommand{\ResearchEntry}[5]{
		\noindent \textbf{#1} \hfill      % Study
		\colorbox{White}{%
			\parbox{10em}{%
			\hfill\color{Black}#2}} \par  % Duration
		\noindent Advisor: \textit{#3} \hfill      % Position
		\colorbox{White}{%
			\parbox{6em}{%
			\hfill\color{Black}#4}} \par  % Location
		\noindent\hangindent=2em\hangafter=0 \small #5 % Description
		\normalsize \par}
		
\newcommand{\ServiceEntry}[4]{
		\noindent \textbf{#1} \hfill      % Study
		\colorbox{White}{%
			\parbox{10em}{%
			\hfill\color{Black}#2}} \par  % Duration
		\noindent \textit{#3} \par  % Location
		\noindent\hangindent=2em\hangafter=0 \small #4 % Description
		\normalsize \par}

\newcommand{\WorkEntry}[4]{				  % Same as \EducationEntry
		\noindent \textbf{#1} \hfill      % Jobname
		\colorbox{White}{\color{White}#2} \par  % Duration
		\noindent \textit{#3} \par              % Company
		\noindent\hangindent=2em\hangafter=0 \small #4 % Description
		\normalsize \par}
        
\newcommand{\OpenSourceEntry}[3]{				  % Same as \EducationEntry
		\noindent \textbf{#1}  \par    % Jobname
		\noindent \url{#2} \par            % Company
		\noindent #3 \par % Description
        }

\newcommand{\PaperEntry}[6]{
		\noindent #1, ``\href{#6}{#2}", \textit{#3}, #4 (#5).}

\newcommand{\ArxivEntry}[3]{
		\noindent #1, ``\href{http://arxiv.org/abs/#3}{#2}", \textit{{cond-mat/}#3}.}
        
\newcommand{\BookEntry}[4]{
		\noindent #1, ``\href{#3}{#4}", \textit{#3}.}
        
\newcommand{\FundingEntry}[5]{
        \noindent #1, ``#2", \$#3 (#4, #5).}

\newcommand{\TalkEntry}[4]{
		\noindent #1, #2, #3 #4}

\newcommand{\CommitteeEntry}[3]{
		\noindent #1 (#2): #3}

\newcommand{\ThesisEntry}[5]{
		\noindent #1 -- #2 #3 ``#4" \textit{#5}}
		
\newcommand{\CourseEntry}[4]{
		\noindent \textbf{#1} \par
		\noindent \textit{#3} \hfill      % Study
		\colorbox{White}{%
			\parbox{6em}{%
			\hfill\color{Black}#2}} \par  % Duration
		\noindent\hangindent=2em\hangafter=0 \small #4 % Description
		\normalsize \par}

\newcommand{\PressEntry}[5]{
		\noindent \textbf{#5}. #1 (#2). \textit{#3} Retrieved from: \url{#4}}

%%% Begin Document
%%% ------------------------------------------------------------
\begin{document}

%\layout

% you can upload a photo and include it here...
%\begin{wrapfigure}{l}{0.5\textwidth}
%	\vspace*{-2em}
%		\includegraphics[width=0.2\textwidth]{Appelbaum.JPG}
%\end{wrapfigure}

\MyName{Natasha Jaques}
\MySlogan{\vspace{-0.1in}
\begin{flushright}
1635 Cedar Street\\
Berkeley, CA 94703 \\
(425) 463-9162\\
\href{mailto:natashamjaques@gmail.com}{natashamjaques@gmail.com}\\
\end{flushright}}

%\sepspace
%\NewPart{Overview}{}
% Professor of Physics, APS Fellow, Condensed Matter Experiment {\em and} Theory.
%%% Personal details
%%% ------------------------------------------------------------
% \NewPart{}{}

% \PersonalEntry{Address}{\href{http://maps.google.com/maps/ms?ie=UTF&msa=0&msid=205373386241137011999.000494c753d05031c4fe4}{Physical Sciences Complex}, University of Maryland, College Park}
% \PersonalEntry{Phone}{(301) 337-7461}
% \PersonalEntry{Mail}{\href{mailto:appelbaum@physics.umd.edu}{appelbaum@physics.umd.edu}}
% \PersonalEntry{WWW}{\href{http://appelbaumlab.umd.edu/CV.html}{http://appelbaum.physics.umd.edu}}

%%% Education
%%% ------------------------------------------------------------
\NewPart{Education}{}

\EducationEntry{Ph.D. Media Arts and Sciences}{2014-2019} 
            {Massachusetts Institute of Technology, GPA: 5.0}%{\href{http://dspace.mit.edu/handle/1721.1/29612}{``Ballistic Electrons: Microscopy, Spectroscopy, Devices and Luminescence"}\\
%Thesis advisor: V. Narayanamurti (Harvard) / J.D. Joannopoulos (MIT)}
\smallskip

\EducationEntry{M.Sc. Computer Science}{2012-2014}{University of British Columbia, GPA: 94.7\%}%{Summa Cum Laude. Academic/Research advisors: X.-C. Zhang and P.D. Persans}
\smallskip

\EducationEntry{B.Sc. Computer Science}{2007-2012}{University of Regina, GPA: 92.8\%}
\smallskip

\EducationEntry{B.A. Psychology}{2007-2012}{University of Regina, GPA: 92.0\%}

\NewPart{Awards}{}
\textbf{Best paper awards:}
\begin{itemize}
\item 2020 Best Paper at the NeurIPS Cooperative AI workshop
\item 2019 Best Paper Honourable Mention at the International Conference on Machine Learning (ICML)
\item 2019 Best Paper Nominee at the NeurIPS Conversational AI workshop
\item 2016 Best Demo at Neural Information Processing Systems (NeurIPS)
\item 2016 Best Paper at the NeurIPS Machine Learning for Healthcare Workshop
\end{itemize}

\noindent \textbf{Grants:}
\begin{itemize}
\item 2021 \href{https://c3dti.ai/c3-announces-energy-climate-awards/}{C3.ai Digital Transformation Institute AI for Energy and Climate Security Awards}, “Offline Reinforcement Learning for Energy-Efficient Power Grids”, \$210,000. % 
\end{itemize}

\noindent \textbf{Merit awards:}
\begin{itemize}
\item 2019 Rising Stars in EECS Pitch Competition Winner, Computer Science
\item 2012 S.E. Stewart Award for the highest GPA in an Arts degree
\item 2011 Faculty of Science Supplemental Instruction Scholar Award
\end{itemize}

\noindent \textbf{Scholarships:}
\begin{itemize}
\item 2014 NSERC Doctoral Postgraduate Fellowship
\item 2014 Robert Wood Johnson Foundation Wellbeing Fellowship
\item 2013 Microsoft Research Graduate Women's Scholarship
\item 2013 UBC Affiliated Fellowships Scholarship
\item 2012 UBC CS Merit Scholarship
\item 2012 NSERC Canadian Graduate Scholarship
\item 2012 ISM Canada IT Futures Scholarship
\item 2011 Shell Canada Scholarship in Computer Science
\item 2010 John \& Jack Spencer Gordon Middleton Scholarship
\item 2007-2012 Academic Gold Scholarship
\end{itemize}

%%% Work experience
%%% ------------------------------------------------------------
\NewPart{Research and Work Experience}{}
\ResearchEntry{Google Brain, Research Scientist \& UC Berkeley, Post-doc}{Berekeley, USA}{Sergey Levine \& Douglas Eck}{2020}
{\begin{itemize}
    \item Social learning helps humans and animals rapidly adapt to new circumstances, and drives the emergence of complex learned behaviors. My research focuses on \textbf{Social Reinforcement Learning}, which asks whether social learning can provide the same benefits to AI agents, while improving human-AI interaction. 
    \item This includes developing \textbf{multi-agent} training algorithms that create an automatic \textbf{curriculum} to aid learning, exploration, and generalization, enabling agents to solve problems that would not otherwise be possible when learning alone.
    \item Improving \textbf{human-AI interaction} through mechanisms that allow agents to learn from, and rapidly adapt to and coordinate with, human partners. 
    \item Creating techniques to enhance multi-agent \textbf{social learning} and \textbf{coordination} with other agents.
\end{itemize}}
\sepspace

\ResearchEntry{MIT Media Lab, PhD Candidate}{Boston, USA}{Rosalind Picard}{2014-2019}
{\begin{itemize}
    \item My PhD focused on both advancing \textbf{Affective Computing} (AC) through the development of sophisticated machine learning methods tailored to AC problems, and improving \textbf{machine learning} via enhanced \textbf{social and emotional} intelligence. 
    \item Focused on improving the generalization of machine learning models, via multi-task learning, \textbf{intrinsic motivation, transfer learning, and learning from human preferences.}
    \item Demonstrated that using \textbf{multi-task learning to personalize} machine learning models to suit an individual can improve the accuracy in predicting mood, health, and stress by $\sim20$\%, leading to state-of-the-art performance on this task.
    \item Focused on a range of \textbf{Affective Computing} projects; e.g. automatically filling in missing sensor data in multimodal inputs, predicting mood, stress, health, sleep, and bonding in conversations, and detecting artifacts in physiological signals.
\end{itemize}}
\sepspace

\ResearchEntry{DeepMind, Research Scientist Intern}{London, UK}{Nando de Freitas}{2018}
{\begin{itemize}
    \item Developed a unified mechanism for improving \textbf{coordination} and \textbf{communication} in multi-agent reinforcement learning (MARL) using an intrinsic social motivation to promote \textbf{causal influence} between agents.
    \item Demonstrated that this led to agents spontaneously developing \textbf{emergent communication}, and could be used to learn more meaningful explicit communication protocols. 
    \item Showed that influence can be computed in a \textbf{fully decentralized} manner using each agent's internal model of other agents, without requiring a centralized controller or access to another agent's rewards.
\end{itemize}}
\sepspace

\ResearchEntry{Google Brain, Research Scientist Intern}{San Francisco, USA}{Douglas Eck}{2017-2018}
{\begin{itemize}
    \item Demonstrated the \textbf{first example} of improving the output of a deep learning model through automatic detection of human \textbf{facial expression} reactions to its output.
    \item \textbf{Led a team of five} in conducting \textbf{HCI} research and building a platform to collect people's facial expression responses to samples from the model. 
    \item Used a combinative of \textbf{generative models (VAE and GAN)} to improve the outputs of Sketch RNN according to facial expression response data. 
\end{itemize}}
\sepspace

\ResearchEntry{Google Brain, Research Scientist Intern}{San Francisco, USA}{Douglas Eck}{2016}
{\begin{itemize}
    \item Developed a novel technique for training recurrent neural networks both on data and using reinforcement learning (RL), by treating the model trained on data as a prior policy and penalizing deviation from this prior using \textbf{KL-Control}.
    \item Provided empirical results showing the technique is able to reduce \textbf{catastrophic forgetting} when doing \textbf{transfer learning} from maximum likelihood to RL. 
    \item Demonstrated the effectiveness of the approach for both \textbf{music generation} and \textbf{drug discovery}.
\end{itemize}}
\sepspace

\ResearchEntry{Microsoft, Program Manager Intern}{Redmond, USA}{Brian Jack}{2014}
{\begin{itemize}
    \item Researched how to handle the transition from IPv4 to IPv6 for the \textbf{Networking} team in \textbf{Azure}.
    \item Designed the \textbf{implementation} and initiated writing code to support the transition.
\end{itemize}}
\sepspace

\ResearchEntry{University of British Columbia, Masters Student}{Vancouver, Canada}{Cristina Conati}{2012-2014}
{\begin{itemize}
    \item Developed machine learning methods for detecting when a student using an \textbf{Intelligent Tutoring System} (ITS) was bored, based on their eye tracking and skin conductance data.
    \item Studied methods for optimizing \textbf{POMDPs} to create an emotionally adaptive ITS. 
    \item Compared the performance of deep neural networks and random forests for \textbf{automatically recognizing sign language} symbols from Microsoft Kinect data. 
\end{itemize}}
\sepspace

\ResearchEntry{University of Regina, Honours Undergraduate Student}{Regina, Canada}{David Gerhard, Howard Hamilton}{2012-2014}
{\begin{itemize}
    \item Compared the effectiveness of \textbf{genetic algorithms} vs. reinforcement learning for solving a simulated maze navigation task. 
    \item Developed a series of \textbf{gesture recognition} methods for use with the Microsoft Kinect, in order to create a gesture-based \textbf{musical instrument} interface.
    \item Created an algorithm to \textbf{automatically recognize swimming strokes} in accelerometer data obtained from waterproof devices worn by swimmers.
\end{itemize}}
\sepspace


%%% Talks
%%% ------------------------------------------------------------
\NewPart{Invited Talks}{}
\begin{itemize}
\item\TalkEntry{University of Regina Department of Computer Science Alumni and Friends Lecture Series }{Virtual}{June 2020}{}
\item\TalkEntry{Berkeley Center for Human-Compatible Artificial Intelligence (CHAI) Workshop}{Virtual}{June 2020}{}
\item\TalkEntry{McGill University AI4Good Lab}{Virtual}{May 2020}{}
\item\TalkEntry{ICLR Social AI Virtual Gathering}{Virtual}{May 2020}{}
\item\TalkEntry{Re-Work Women In AI Podcast}{Virtual}{March 2020}{}
\item\TalkEntry{Berkeley Center for Human-Compatible Artificial Intelligence (CHAI) Seminar}{Virtual}{February 2020}{}
\item\TalkEntry{Oxford Department of Computer Science Seminar}{Virtual}{February 2020}{}
\item\TalkEntry{Deep Learning 2.0 Summit}{Virtual}{January 2020}{}
\item\TalkEntry{Institute of Cognitive Science, University of Osnabrück Deep Reinforcement Learning Workshop}{Virtual}{January 2020}{}
\item\TalkEntry{University College London Deciding, Acting, and Reasoning with Knowledge (DARK) Seminar}{Virtual}{December 2020}{}
\item\TalkEntry{Google Apprenticeship Learning Summit}{Virtual}{December 2020}{}
\item\TalkEntry{Microsoft Research New York}{Virtual}{November 2020}{}
\item\TalkEntry{Microsoft AI Breakthroughs Workshop}{Virtual}{September 2020}{}
\item\TalkEntry{DeepMind}{Virtual}{August 2020}{}
\item\TalkEntry{Synthetic Characters Conference}{Virtual}{August 2020}{} 
\item\TalkEntry{Brain RL Seminar}{Virtual}{August 2020}{} 
\item\TalkEntry{Samsung Forum}{Virtual}{April 2020}{} 
\item\TalkEntry{AAAI workshop on Affective Content Analysis (\textbf{Keynote})}{New York, USA}{February 2020}{} 
\item\TalkEntry{Berkeley Center for Human-compatible AI (CHAI) seminar}{Berkeley, USA}{November 2019}{} 
\item\TalkEntry{IBM K-12 Education Conference}{Cambridge, USA}{October 2019}{} 
\item\TalkEntry{ICML workshop on Imitation, Intent, and Interaction}{Cambridge, USA}{June 2019}{} 
\item\TalkEntry{Broad Institute}{Cambridge, USA}{February 2019}{} 
\item\TalkEntry{Schlumberger-Doll Research Center}{Cambridge, USA}{December 2018}{} 
\item\TalkEntry{Google AI}{Montreal, Canada}{December 2018}{} 
\item\TalkEntry{New York University}{New York, USA}{November 2018}{} 
\item\TalkEntry{Starsconf \textbf{(Keynote)}}{Santiago, Chile}{November 2018}{} 

\item\TalkEntry{Creative AI Meetup}{London, UK}{September 2018}{} %{Learning via Social Awareness: Improving a Deep Generative Sketching Model with Facial expression Feedback}

\item\TalkEntry{%Department of Computer Science Image and Video Computing Seminar Series, 
Boston University}{Boston, USA}{March 2017}{} %{Affective Computing}

\item\TalkEntry{%Department of Computer Science Seminar Series, 
University of Regina}{Regina, Canada}{January 2017}{} %{Advances in Deep Learning for Music Generation and Affective Computing}
\end{itemize}

% PRESS
\NewPart{Press}{}
\begin{itemize}
\item\PressEntry{Hutson, M.}{2021, January 19}{Who needs a teacher? Artificial intelligence designs lesson plans for itself.}{https://www.sciencemag.org/news/2021/01/who-needs-teacher-artificial-intelligence-designs-lesson-plans-itself}{Science}

\item\PressEntry{Hutson, M.}{2019, June 17}{DeepMind Teaches AI Teamwork.}{https://spectrum.ieee.org/tech-talk/computing/software/deepmind-teaches-ai-teamwork}{IEEE Spectrum}

\item\PressEntry{Hao, K.}{2019, June 20}{Here are 10 ways AI could help fight climate change.}{https://www.technologyreview.com/s/613838/ai-climate-change-machine-learning/}{MIT Technology Review}

\item\PressEntry{Snow, J.}{2019, July 18}{How artificial intelligence can tackle climate change.}{https://www.nationalgeographic.com/environment/2019/07/artificial-intelligence-climate-change/}{National Geographic}

\item\PressEntry{Gershgorn, D.}{2018, February 16}{Google is building AI to make humans smile.}{https://qz.com/1209466/google-is-building-ai-to-make-humans-smile/}{Quartz}

\item\PressEntry{Knight, W.}{2016, November 30}{AI songsmith cranks out surprisingly catchy tunes.}{https://www.technologyreview.com/s/603003/ai-songsmith-cranks-out-surprisingly-catchy-tunes/}{MIT Technology Review}

\item\PressEntry{Annear, S.}{2015, January 5}{Website tracks your happiness to remind you life’s not so bad.}{http://www.bostonmagazine.com/news/blog/2015/01/05/smiletracker-captures-photos-internet/}{Boston Magazine}

\end{itemize}


%%% Papers
%%% ------------------------------------------------------------

\NewPart{Publications}{\href{https://scholar.google.co.uk/citations?user=8iCb2TwAAAAJ&hl=en}{[Google Scholar]}}
*Equal Contribution
\begin{etaremune}

\item \PaperEntry{Fickinger, A.*, \underline{Jaques, N.*}, Parajuli, S., Chang, M., Rhinehart, N., Berseth, G., Russell, S., \& Levine, S.}{Explore and Control with Adversarial Surprise}{Neural Information Processing Systems (NeurIPS)}{(submitted)}{2021}{}

\item \PaperEntry{Gur, I., \underline{Jaques, N.}, Malta, K., Tiwari, M., Lee, H., \& Faust, A.}{Adversarial Environment Generation for Learning to Navigate the Web}{Neural Information Processing Systems (NeurIPS)}{(submitted)}{2021}{https://arxiv.org/abs/2103.019917}

\item \PaperEntry{Filos, A., Lyle, C., Gal, Y., Levine, S., \underline{Jaques, N.*}, \& Farquhar, G.*}{PsiPhi-Learning: Reinforcement Learning with Demonstrations using Successor Features and Inverse Temporal Difference Learning}{International Conference on Machine Learning (ICML) \textbf{Long talk (top 3\% of submissions)}}{Virtual}{2021}{https://arxiv.org/pdf/2102.12560.pdf}

\item \PaperEntry{Ndousse, K., Eck, D., Levine, S., \& \underline{Jaques, N.}}{Emergent Social Learning via Multi-agent Reinforcement Learning}{International Conference on Machine Learning (ICML) }{Virtual}{2021}{https://arxiv.org/abs/2010.00581}

\item \PaperEntry{Lee, D., \underline{Jaques, N.}, Kew, J., Eck, D., Schuurmans, D., \& Faust, A.}{Joint Attention for Multi-Agent Coordination and Social Learning}{ICRA Social Intelligence Workshop \textbf{Spotlight talk}}{Virtual}{2021}{https://arxiv.org/abs/2104.07750}

\item \PaperEntry{Ndousse, K., Eck, D., Levine, S., \& \underline{Jaques, N.}}{Learning Social Learning}{NeurIPS Cooperative AI Workshop \textbf{Best Paper} }{Virtual}{2020}{https://arxiv.org/abs/2010.00581}

\item \PaperEntry{Dennis, M.*, \underline{Jaques, N.*}, Vinitsky, E., Bayen, A., Russell, S., Critch, A., \& Levine, S.}{Emergent Complexity and Zero-Shot Transfer via Unsupervised Environment Design}{Neural Information Processing Systems (NeurIPS) \textbf{Oral (top 1\% of submissions)}}{Virtual}{2020}{https://arxiv.org/abs/2012.02096}

\item \PaperEntry{\underline{Jaques, N.*}, Shen, J. H.*, Ghandeharioun, A., Ferguson, C., Lapedriza, A., Jones, N., Gu, S., \& Picard, R.}{Human-Centric Dialog Training via Offline Reinforcement Learning}{Empirical Methods in Natural Language Processing (EMNLP)}{Virtual}{2020}{https://arxiv.org/pdf/2010.05848.pdf}

\item \PaperEntry{\underline{Jaques, N.}}{Social and Affective Machine Learning}{Massachusetts Institute of Technology}{PhD Thesis}{2019}{https://www.media.mit.edu/publications/social-and-affective-machine-learning/}

\item \PaperEntry{Saleh, A.*, \underline{Jaques, N.*}, Ghandeharioun, A., Shen, J. H., \& Picard, R.}{Hierarchical Reinforcement Learning for Open-Domain Dialog}{Association for the Advancement of Artificial Intelligence (AAAI) \textbf{Oral (top 7.8\% of submissions)}}{New York, USA}{2019}{https://arxiv.org/abs/1909.07547}

\item \PaperEntry{\underline{Jaques, N.}, Ghandeharioun, A., Shen, J. H., Ferguson, C., Lapedriza, A., Jones, N., Gu, S., \& Picard, R.}{Way Off-Policy Batch Deep Reinforcement Learning of Implicit Human Preferences in Dialog}{Neural Information Processing Systems (NeurIPS) Workshop on Conversational AI}{Vancouver, Canada}{2019}{https://arxiv.org/abs/1907.00456}

\item \PaperEntry{Ghandeharioun, A.*, Shen, J. H.*, \underline{Jaques, N.*}, Ferguson, C., Jones, N., Lapedriza, A., \& Picard, R.}{Approximating Interactive Human Evaluation with Self-Play for Open-Domain Dialog Systems}{Neural Information Processing Systems (NeurIPS)}{Vancouver, Canada}{2019}{https://arxiv.org/abs/1906.09308}

\item \PaperEntry{Rolnick, D., Donti, P. L., Kaack, L. H., Kochanski, K., Lacoste, A., Sankaran, K., Ross, A. S., Milojevic-Dupont, N., \underline{Jaques, N.}, Waldman-Brown, A., Luccioni, A., Maharaj, T., Sherwin, E. D., Mukkavilli, S. K., Kording, K. P., Gomes, C., Ng, A. Y., Hassabis, D., Platt, J. C., Creutzig, F., Chayes, J., Bengio, Y.}{Tackling Climate Change with Machine Learning}{(Arxiv preprint)}{}{2019}{https://arxiv.org/abs/1906.05433}

\item \PaperEntry{\underline{Jaques, N.}, Lazaridou, A., Hughes, E., Gulcehre, C., Ortega, P. A., Strouse, D. J., Leibo, J.Z. \& de Freitas, N.}{Social Influence as Intrinsic Motivation for Multi-Agent Deep Reinforcement Learning}{International Conference on Machine Learning (ICML) \textbf{Best Paper Honourable Mention (top 0.26\% of submissions)}}{Long Beach, USA}{2019}{https://arxiv.org/pdf/1810.08647.pdf}

\item \PaperEntry{Jones, N., \underline{Jaques, N.}, Pataranutaporn, P., Ghandeharioun, A., \& Picard, R.}{Automatic Triage and Analysis of Online Suicide Risk with Document Embeddings and Latent Dirichlet Allocation}{Affective Computing and Intelligence Interaction (ACII) workshop on Machine Learning for Mental Health}{}{2019}{https://drive.google.com/file/d/1muoFj_BXJUZCRyjCLEX9DxKOz7b9nKtj/view?usp=sharing}

\item \PaperEntry{\underline{Jaques, N.}, McCleary, J., Engel, J., Ha, D., Bertsch, F.,  Eck, D. \& Picard, R.}{Learning via Social Awareness: Improving a Deep Generative Sketching Model with Facial Feedback}{International Conference on Learning Representations (ICLR) workshop}{}{2018}{https://arxiv.org/pdf/1802.04877.pdf}

\item \PaperEntry{Johnson, K., Taylor, S., Fedor, S., \underline{Jaques, N.}, Chen, W., \& Picard, R.}{Vomit Comet Physiology: Autonomic Changes in Novice Flyers}{IEEE Engineering in Medicine and Biology Society (EMBC)}{Honolulu, USA}{2018}{https://dspace.mit.edu/bitstream/handle/1721.1/123805/18.Johnson-etal_EMBC18_VomitComet.pdf?sequence=1&isAllowed=y}

\item \PaperEntry{\underline{Jaques, N.}, Gu, S., Bahdanau, D., Hern\'{a}ndez-Lobato, J. M., Turner, R. E. \& Eck, D.}{Sequence Tutor: Conservative Fine-Tuning of Sequence Generation Models with KL-control}{International Conference on Machine Learning (ICML)}{Sydney, Australia}{2017}{https://arxiv.org/pdf/1611.02796.pdf}

\item \PaperEntry{Taylor, S.*, \underline{Jaques, N.*}, Nosakhare, E., Sano, A. \& Picard, R.}{Personalized Multitask Learning for Predicting Tomorrow's Mood, Stress, and Health}{IEEE Transactions on Affective Computing}{}{2017}{https://affect.media.mit.edu/pdfs/17.TaylorJaques-PredictingTomorrowsMoods.pdf}

\item \PaperEntry{\underline{Jaques, N.}, Rudovic, O., Taylor, S., Sano, A. \& Picard, R.}{Predicting Tomorrow’s Mood, Health, and Stress Level using Personalized Multitask Learning and Domain Adaptation}{Proceedings of Machine Learning Research}{48, 17-33}{2017}{http://proceedings.mlr.press/v66/jaques17a/jaques17a.pdf}

\item \PaperEntry{\underline{Jaques, N.}, Taylor, S., Sano, A. \& Picard, R.}{Multimodal Autoencoder: A Deep Learning Approach to Filling in Missing Sensor Data and Enabling Better Mood Prediction}{International Conference on Affective Computing and Intelligent Interaction (ACII)}{San Antonio, USA}{2017}{https://affect.media.mit.edu/pdfs/17.Jaques_autoencoder_ACII.pdf}

\item \PaperEntry{Taylor, S., \underline{Jaques, N.}, E. Nosakhare, Sano, A., Klerman, E. B. \& Picard, R.}{Importance of Sleep Data in Predicting Next-Day Stress, Happiness, and Health in College Students}{Journal of Sleep and Sleep Disorders Research (suppl\_1)}{A294-A295}{2017}{https://affect.media.mit.edu/pdfs/17.Taylor-etal-MoodPrediction-SLEEP2017-Poster.pdf}

\item \PaperEntry{\underline{Jaques, N.}, Gu, S., Turner, R. E. \& Eck, D.}{Tuning Recurrent Neural Networks with Reinforcement Learning}{International Conference on Learning Representations (ICLR) - workshop}{Toulon, France}{2016}{https://openreview.net/pdf?id=Syyv2e-Kx}

\item \PaperEntry{\underline{Jaques, N.*}, Taylor S.*, Nosakhare E., Sano A. \& Picard R.}{Multi-task Learning for Predicting Health, Stress, and Happiness}{Neural Information Processing Systems (NeurIPS) Workshop on Machine Learning for Healthcare \textbf{Best Paper}}{Barcelona, Spain}{2016}{https://pdfs.semanticscholar.org/b228/7a406985980515d5cc63e9b37fb17c5186f8.pdf}

\item \PaperEntry{Roberts, A., Engel, J., Hawthorne, C., Simon, I., Waite, E., Oore, S., \underline{Jaques, N.}, Resnick, C. \& Eck, D.}{Interactive Musical Improvisation with Magenta}{Neural Information Processing Systems (NeurIPS) \textbf{Best Demo}}{Barcelona, Spain}{2016}{https://nips.cc/Conferences/2016/Schedule?showEvent=6307}

\item \PaperEntry{\underline{Jaques, N.}, McDuff, D., Kim, Y. K. \& Picard R.}{Understanding and Predicting Bonding in Conversations Using Thin Slices of Facial Expressions and Body Language}{Intelligent Virtual Agents (IVA)}{Los Angeles, USA}{2016}{http://affect.media.mit.edu/pdfs/16.Jaques-IVAbonding.pdf}

\item \PaperEntry{\underline{Jaques, N.}, Kim, Y. K. \& Picard R.}{Personality, Attitudes, and Bonding in Conversations}{Intelligent Virtual Agents (IVA)}{Los Angeles, USA}{2016}{http://affect.media.mit.edu/pdfs/16.Jaques-IVApersonality.pdf}

\item \PaperEntry{\underline{Jaques, N.}, Rich, T., Dinakar, K., Farve, N., Chen, W.V., Maes, P. \& Picard, R.}{BITxBIT: Encouraging Behavior Change with N=2 Experiments}{Proceedings of the CHI Conference Extended Abstracts on Human Factors}{San Jose, USA}{2016}{http://affect.media.mit.edu/pdfs/16.Jaques-etal-CHI.pdf}

\item \PaperEntry{Taylor, S., \underline{Jaques, N.}, Sano, A., Azaria, A., Ghandeharioun, A. \& Picard, R.}{Machine Learning of Sleep and Wake Behaviors to Classify Self-Reported Evening Mood}{Sleep}{Denver, USA}{2016}{https://affect.media.mit.edu/pdfs/16.Taylor-ClassifyingSelfReportedMood-SLEEP2016.pdf}

\item \PaperEntry{\underline{Jaques, N.*}, Taylor, S.*, Sano, A. \& Picard, R.}{Multi-task Multi-Kernel Learning for Estimating Individual Wellbeing}{Neural Information Processing Systems (NeurIPS) Workshop on Multimodal Machine Learning}{Montreal, Canada}{2015}{https://affect.media.mit.edu/pdfs/15.Jaques-etal-NIPSMMML.pdf}

\item \PaperEntry{Xia, V., \underline{Jaques, N.}, Taylor, S., Fedor, S. \& Picard, R.}{Active learning for Electrodermal Activity classification}{IEEE Conference on Signal Processing in Medicine and Biology (SPMB)}{Philadelphia, USA}{2015}{https://dspace.mit.edu/openaccess-disseminate/1721.1/109392}

\item \PaperEntry{\underline{Jaques, N.*}, Taylor, S.*, Azaria, A., Ghandeharioun, A., Sano, A. \& Picard R.}{Predicting students' happiness from physiology, phone, mobility, and behavioral data}{International Conference on Affective Computing and Intelligent Interaction (ACII)}{Xi'an, China}{2015}{https://www.ncbi.nlm.nih.gov/pmc/articles/PMC5431070/}

\item \PaperEntry{Taylor, S.*, \underline{Jaques, N.*}, Chen, W., Fedor, S., Sano, A. \& Picard, R.}{Automatic identification of artifacts in Electrodermal Activity data}{International Conference of the IEEE Engineering in Medicine and Biology Society (EMBC)}{Milan, Italy}{2015}{https://www.ncbi.nlm.nih.gov/pmc/articles/PMC5413200/}

\item \PaperEntry{Chen, W., \underline{Jaques, N.}, Taylor, S., Sano, A., Fedor, S. \& Picard R.}{Wavelet-based motion artifact removal for Electrodermal Activity}{International Conference of the IEEE Engineering in Medicine and Biology Society (EMBC)}{Milan, Italy}{2015}{https://www.ncbi.nlm.nih.gov/pmc/articles/PMC5413204/}

\item \PaperEntry{Sano, A., Yu, A.Z., McHill, A.W., Phillips, A.J., Taylor, S., \underline{Jaques, N.}, Czeisler, C. A., Klerman, E. B. \& Picard, R.}{Prediction of happy-sad mood from daily behaviors and previous sleep history}{International Conference of the IEEE Engineering in Medicine and Biology Society (EMBC)}{Milan, Italy}{2015}{https://www.ncbi.nlm.nih.gov/pmc/articles/PMC4768795/}

\item \PaperEntry{\underline{Jaques, N.} \& Farve, N.}{Engaging the workplace with challenges}{International Conference on Persuasive Technologies}{Chicago, USA}{2015}{https://drive.google.com/file/d/15Pimm1FwwxSPPX04KTr84ilVHJvVExPd/view?usp=sharing}

\item \PaperEntry{\underline{Jaques, N.}, Chen, W. V. \& Picard, R.}{SmileTracker: Automatically and Unobtrusively Recording Smiles and their Context.}{Proceedings of the CHI Conference Extended Abstracts}{Seoul, Korea}{2015}{https://affect.media.mit.edu/pdfs/15.jaques-chen-picard-CHI.pdf}

\item \PaperEntry{Sano, A., Phillips, A. J., Yu, A. Z., McHill, A. W., Taylor, S., \underline{Jaques, N.}, Czeisler, C. A., Klerman, E. B. \& Picard, R.}{Recognizing academic performance, sleep quality, stress level, and mental health using personality traits, wearable sensors and mobile phones}{Wearable and Implantable Body Sensor Networks (BSN)}{Cambridge, USA}{2015}{https://www.ncbi.nlm.nih.gov/pmc/articles/PMC5431072/}

\item \PaperEntry{\underline{Jaques, N.}}{Predicting Affect in an Intelligent Tutoring System}{University of British Columbia}{Master's Thesis}{2014}{https://open.library.ubc.ca/collections/ubctheses/24/items/1.0135541}

\item \PaperEntry{\underline{Jaques, N.}, Conati, C., Harley, J. M. \& Azevedo, R.}{Predicting Affect from Gaze Data During Interaction with an Intelligent Tutoring System}{Intelligent Tutoring Systems}{Honolulu, USA}{2014}{http://www.cs.ubc.ca/~conati/my-papers/ITS-Natasha-2014.pdf}

\item \PaperEntry{Conati, C., \underline{Jaques, N.} \& Muir, M.}{Understanding attention to adaptive hints in educational games: an eye-tracking study}{International Journal of Artificial Intelligence in Education}{23(1-4), 136-161}{2013}{https://link.springer.com/article/10.1007/s40593-013-0002-8}

\item \PaperEntry{\underline{Jaques, N.}}{Emotionally Adaptive Intelligent Tutoring Systems using POMDPs}{Unpublished manuscript}{}{2013}{http://www.cs.ubc.ca/~jaquesn/POMDPPaper.pdf}

\item \PaperEntry{\underline{Jaques, N.}}{Fast Johnson–Lindenstrauss transform for classification of high dimensional data}{Unpublished manuscript}{}{2013}{http://www.cs.ubc.ca/~jaquesn/MachineLearningTheory.pdf}

\item \PaperEntry{\underline{Jaques, N.} \& Nutini, J.}{A Comparison of Random Forests and Dropout Nets for Sign Language Recognition with the Kinect}{Unpublished manuscript}{}{2013}{http://www.cs.ubc.ca/~jaquesn/MachineLearningProject.pdf}

\end{etaremune}

% TEACHING
\NewPart{Teaching and Mentorship}{}
% \textit{\textattachfile[color=1 0 1]{teaching.pdf}{@ U. Maryland, Physics Dept.}:}
%\vspace{-24pt}
\CourseEntry{Google Computer Science Research Mentorship Program}{Spring 2021}{Mentor}
{I supported computer science graduate students from under-represented groups through a twelve week program in which I counseled them on research, and provided career and networking advice.}
\sepspace

\CourseEntry{NeurIPS Affinity Groups Mentorship}{Winter 2020}{Mentor}
{I participated in Black in AI and Women in ML (WIML) as a mentor, hosting roundtable sessions in which I answered questions and provided guidance about finding a job in industry and choosing between industry and academia.}
\sepspace

\CourseEntry{Richard Tapia Diversity in Computing Conference}{Fall 2020}{Mentor}
{I participated in the conference as a mentor, and hosted a roundtable session with students, answering questions and providing career guidance}
\sepspace

\CourseEntry{NeurIPS Climate Change AI Workshop}{Summer 2020}{Mentor}
{The goal of the CCAI mentorship program is to pair individuals with machine learning research expertise with students working on climate-focused ML projects. I provided research advice related to reinforcement learning to students working on the problem of directing resources to stop wildfire spread.}
\sepspace

\CourseEntry{OpenAI Scholars Program}{Summer 2018, Spring 2020}{Mentor}
{The Scholars program is an initiative to encourage individuals from underrepresented groups to study deep learning. Mentors provide guidance to participants in gaining a technical understanding of deep learning and advise on their projects.}
\sepspace

\CourseEntry{Berkeley AI Research (BAIR) mentoring}{2020}{Mentor}
{As a BAIR mentor, I helped support undergraduates from traditionally underrepresented groups to embark on a career in machine learning research.}
\sepspace

\CourseEntry{MIT MAS.S61 - Personalized Machine Learning (Graduate level)}{Spring 2017}{Teaching Assistant}
{The first iteration of a course focused on multi-task and domain adaptation methods for training machine learning classifiers that are personalized to fit the individual, but which still gain statistical strength from the population. Several projects I completed during my PhD were used to create material for the course; I also delivered lectures, recitations, and helped grade projects.}
\sepspace

\CourseEntry{MIT 6.867 - Machine Learning (Graduate level)}{Fall 2016}{Teaching Assistant}
{MIT's main graduate level course on machine learning methods. In addition to delivering recitations and grading homework and exams, I developed the first deep learning assignment for the course.}
\sepspace

\CourseEntry{UBC CPSC 344 - Intro to HCI Methods (Graduate level)}{Fall 2013}{Teaching Assistant}
{Tools and techniques teaching a systematic approach to interface design, task analysis, analytic and empirical evaluation.}
\sepspace

\CourseEntry{UBC CPSC 430 - Computers and Society}{Fall 2012}{Teaching Assistant}
{A course examining the social and ethical issues of modern technology.}
\sepspace

\CourseEntry{U. of R. CS 102 - Intro to Computer Science}{Summer 2012}{Sessional Lecturer}
{Taught an intro CS course, and developed a novel set of course material, including the syllabus, lecture notes, and homework assignments.}
\sepspace

\CourseEntry{U. of R. Math 110 - Calculus I}{Fall 2010}{Supplemental Instruction}
{Delivered recitations to help students work understand differentiation, integration, optimization, etc.}
\sepspace

\CourseEntry{U. of R. CS 102 - Intro to Computer Science}{2008-2012}{Supplemental Instruction}
{Led optional recitations for intro to computer science, answering questions and designing problems to help students understand the material. As one of the first students to participate in the program, I helped to develop materials for it, including creating a web app to support it.}


\iffalse
%Grants and Fellowships (if you are in a field where these differ categorically from Awards and Honors). Give funder, institutional location in which received/utilized, year span. Listing $$ amount appears to be field-specific.  Check with a trusted senior advisor. Year at left.
MIT Machine Learning Across Disciplines Challenge

\fi

\NewPart{Service to Profession}{}
\begin{itemize}
    \item \CommitteeEntry{Neural Information Processing Systems (NeurIPS)}{2019, 2020, 2021}{Reviewer (Top ranked)}
    \item \CommitteeEntry{International Conference on Machine Learning (ICML)}{2019, 2020, 2021}{Reviewer}
    \item \CommitteeEntry{International Conference on Learning Representations (ICLR)}{2021}{Reviewer}
    \item \CommitteeEntry{Affective Computing and Intelligent Interaction (ACII)}{2021}{Reviewer}
    \item \CommitteeEntry{Workshop on Cooperative AI at NeurIPS)}{2020}{Program Committee, \textbf{Panelist, Q\&A Moderator}}
    \item \CommitteeEntry{Workshop on Offline RL at NeurIPS)}{2020}{Program Committee}
    \item \CommitteeEntry{Workshop on Emergent Communication (EmeComm) at NeurIPS}{2019}{\textbf{Organizer}}
    \item \CommitteeEntry{Workshop on Climate Change for Artificial Intelligence (CCAI) at NeurIPS}{2019}{\textbf{Organizer}}
    \item \CommitteeEntry{Women in Machine Learning (WIML) at NeurIPS}{2019}{Reviewer}
    \item \CommitteeEntry{Association for the Advancement of Artificial Intelligence (AAAI)}{2019, 2020}{Reviewer}
    \item \CommitteeEntry{Imitation, Intent, and Interaction Workshop (IIIW) at ICML}{2019}{Program Committee}
    \item \CommitteeEntry{Transactions on Affective Computing (TAFFC)}{2019}{Reviewer}
    \item \CommitteeEntry{Transactions on Audio, Speech, and Language Processing (TASL)}{2019}{Reviewer}
    \item \CommitteeEntry{Workshop on Artificial Intelligence in Affective Computing (AffComp) at ICML}{2018}{\textbf{Organizer}}
    \item \CommitteeEntry{Workshop on Machine Learning for Healthcare (ML4HC) at NeurIPS}{2018}{Program Committee}
    \item \CommitteeEntry{Workshop on Artificial Intelligence in Affective Computing (AffComp) at IJCAI}{2018}{Program Committee}
    \item \CommitteeEntry{Transactions on Computer-Human Interaction (ToCHI)}{2017}{Reviewer}
    \item \CommitteeEntry{Transactions on Knowledge and Data Engineering (TKDE)}{2017}{Reviewer}
    \item \CommitteeEntry{Workshop on Machine Learning for Healthcare (ML4HC) at NeurIPS}{2016, 2017}{Program Committee}
    \item \CommitteeEntry{Computer Human Interaction (CHI)}{2015, 2016}{Reviewer}

\end{itemize}

\NewPart{Community Service}{}

% \noindent \textbf{Professor of Physics} \hfill      % Study
% 		\colorbox{White}{%
% 			\parbox{6em}{%
% 			\hfill\color{Black}2016-$\qquad$}} \par

% \EducationEntry{{Ph.D. Candidate}}{2014-$\qquad$}{University of Maryland, College Park}{
% }
\ServiceEntry{Climate Change AI (CCAI) mentor}{2021}{Virtual}
{I served as a Research Mentor as part of the Climate Change AI (CCAI) ICML workshop program. I provided feedback, advice, and research discussions to a mentee interested in submitting to the workshop}
\sepspace

\ServiceEntry{Climate Change AI (CCAI)}{2020}{Virtual}
{As one of the committee members of CCAI (\url{https://www.climatechange.ai/}), I helped catalyze the use of machine learning technology for mitigating and adapting to climate change through building a community of diverse stakeholders and connecting them to resources such as information and funding.}
\sepspace

\ServiceEntry{Media Lab Carbon Offsets}{2019}{Cambridge, USA}
{Together with two other students, I petitioned for funding to create a program to offset the carbon emissions of research-related flights for the Media Lab: \url{https://offset.media.mit.edu/}}
\sepspace

\ServiceEntry{Students Offering Support}{2015, 2017}{Cambridge, USA}
{SOS aims to assist under-represented students applying to the Media Lab. I helped participants with their application portfolio and statement.}
\sepspace

\ServiceEntry{Cradles to Crayons}{2015}{Boston, USA}
{I helped to sort and package donations of clothing, toys, and books for children in low-income families.}
\sepspace

\ServiceEntry{UBC Thunderbots}{2013-2014}{Vancouver, Canada}
{Thunderbots is a project to build competitive, soccer playing robots. I contributed by working on the AI driving the robots' plays and tactics}
\sepspace

\ServiceEntry{Girlsmarts}{2012-2014}{Vancouver, Canada}
{Planned and organized workshops aimed at increasing interest in Computer Science among elementary-school-aged girls. Helped teach programming and robotics activities.}
\sepspace

\ServiceEntry{UBC CS Graduate Student Assocation - Vice President, Social }{2013}{Vancouver, Canada}
{I organized social events for the graduate Computer Science students at UBC.}
\sepspace

\ServiceEntry{U. of R. CS Students' Society - Vice President}{2010-2012}{Regina, Canada}
{I participated in running the Computer Science students' society, including organizing fundraising events.}
\sepspace

\ServiceEntry{Mother Teresa Middle School Science Camp}{2011}{Regina, Canada}
{I volunteered for an \textit{option gratuis} summer camp for economically and socially disadvantaged youth from the core of Regina. We built robots.}
\sepspace

\NewPart{Languages and Tools}{}
Python, Tensorflow, Pytorch, Ruby, Django, Javascript, Matlab, C, C$++$, C\#, Scipy, Scikit-learn, Pandas AWS, Google App Engine, Theano, Weka, Angular, Git, SVN, PHP, VB, Octave, Prolog

\NewPart{Open Source Projects}{}
\OpenSourceEntry{Social RL}{https://github.com/google-research/google-research/tree/master/social\_rl}{The Social RL repo comprises three different repos. The first is a set of multi-agent training environments including navigation and sparse reward tasks. Second is code to train multiple independent agents using TF Agents and PPO. Finally, the third repo supports adversarial environment generation, which enables learning an RL policy to generate environments in order to challenge a second set of learning agents.}
\sepspace

\OpenSourceEntry{EDA Explorer}{http://eda-explorer.media.mit.edu/}{An open source web application which allows anyone to upload Electrodermal Activity (EDA) data, visualize it, and analyze it using built-in machine learning algorithms. The site currently has hundreds of users, hosts thousands of EDA files, and has accelerated a number of research projects.}
\sepspace

\OpenSourceEntry{Sequence Tutor}{https://github.com/tensorflow/magenta/tree/master/magenta/models/rl_tuner}{Code for training an RNN sequence model first on data, and then with a reinforcement learning loss function combining temporal difference learning with extrinsic reward, while penalizing KL-divergence from the policy of the pre-trained model.}
\sepspace

\OpenSourceEntry{Personalized Multi-task Learning}{https://github.com/mitmedialab/PersonalizedMultitaskLearning}{Code for three types of multi-task learning (MTL) models personalized to individual human data. Models include: i) a non-parametric hierarchical Bayesian model, ii) a MTL deep neural network, iii) Multi-task Multiple Kernel Learning, and all relevant baselines.}



\end{document}
